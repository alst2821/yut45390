\documentstyle[a4,12pt]{article}
\begin{document}


{\bf Some suggestions for postdoctoral opportunities in Mathematics}

\smallskip

{\bf EPSRC} (www.epsrc.ac.uk)

\smallskip

 EPSRC Postdoctoral Fellowships in Mathematical Sciences. Last up to 3 years. 7--10
 awarded each year, from around 70 applications. 
Next deadline  24 Sept 2004. Need a referee from outside home institution. 
EPSRC likes people to move around. Applicants of any nationality can apply. Emphasise 
that the research is not just a continuation of PhD.

\smallskip

EPSRC Research Assistantships. For these, a more senior mathematican applies for a research grant
to fund a Research Assistant (and provide funding for other visitors, travel, etc.)
 The application can name you
as the RA, or you may see the RA-ship advertised. Less independent -- there is some 
obligation to work in the area of the grant.

\smallskip

EPSRC Advanced Fellowships. Very attractive positions, but not for people straight out of PhD
(candidates should have 3--10 years postdoctoral experience). About 40 awarded per year.
 Last for 5 years.

\smallskip

Some other European countries will have analogous schemes.

\medskip


{\bf Marie-Curie} (http://europa.eu.int/comm/research/fp6/mariecurie-actions, or
http://mc-opportunities.cordis.lu)

\smallskip

Marie-Curie Intra-European Fellowships. Next deadline early 2005?? Typically 2-year postdocs.
Strict mobility requirements (if you've done PhD in UK, you probably can't take the postdoc in
either UK or your own country). Long application form, takes a lot of preparation, and coordination 
with host organisation. Emphasise `complementary training'. Application assessed partly on 
strength of host. 
Referees may not know anything of your area. Restriction on nationality of applicants, but there's 
another scheme `Incoming International Fellowships' for applicants from other countries. Also a scheme
`Outgoing International Fellowships' for Europeans to take a postdoc outside Europe. 


\smallskip

Other Marie-Curie:  There are several Marie-Curie Research Training Networks which provide 
funding for postdocs 
(same mobility requirements). These may be easier to get. Consider applying to RTNs not in your
 immediate area but in related area. Also try Marie-Curie Excellence Grants (similar to RTNs).
See above cordis website for info.

\medskip

{\bf Postdoctoral Fellowships in departments, colleges}


 Junior Research Fellowships at Oxford and Cambridge colleges. Many colleges advertise these,
 some in groupings.
Some are restricted to science, or arts, one or two (eg the GCHQ-funded one at Merton College, Oxford)
 is restricted to Mathematics.
Deadlines mostly in the Autumn. Very competitive. See
http://www.admin.cam.ac.uk/reporter/2003-04/weekly/
for some listings of Oxford and Cambridge JRFs.

\smallskip

A few other UK mathematics departments (Imperial, Edinburgh...) have postdoctoral positions. 
Sometimes these
 can be linked to Leverhulme
awards.

\medskip


{\bf Royal Commission for the Exhibition of 1851 Awards.}  About 6 awards,
 possibly engineering-focussed? See
http://www.royalcommission1851.org.uk

\medskip



{\bf Royal Society} (www.royalsoc.ac.uk)

\smallskip

Dorothy Hodgkin Fellowships. Last deadline was 6 Feb 2004. About 10 awarded per year. These 
used to be
awarded only to women. This no longer applies, but they appear to be designed with great
 career flexibility in mind
(part-time, career breaks). Max 4 years. Any nationality, but there should be some UK connection. 

\smallskip

University Research Fellowships. Must have between 2 and 7 years postdoc experience
 (a bit like EPSRC Advanced Fellowships).

For other Royal Society Fellowships (eg Industry Fellowships) see website.

\medskip


{\bf Other}
Many  strong research departmenst in the USA and Canada (and probably many other countries)
have their own advertised postdoctoral positions. Look on departmental websites, or write
 to individuals with whom you have had research contact, to see if they 
know of funding. In USA and Canada, postdoctoral positions often involve significant 
lecturing to undergraduates, so in applications
it is important to emphasise teaching experience.

\medskip

{\bf NSF Graduate Training Sites in USA}  There are several NSF-funded training 
sites in the USA, in specific research areas,
which my be able to provide postdoctoral funding for short periods. Try\\
http://www.nsf.gov/home/crssprgm/igert/igertprojects.htm

\medskip

{\bf European Postdoctoral Institute for Mathematical Sciences (EPDI)} This 
provides grants of up to 2 years, open to European citizens (European Union,
associated countries and Eastern European countries) for post-docs in
mathematics or theoretical physics. There are 4,5 a year, with the
requirement for the fellows to spend 6 to 12 month visits at several EPDI
institutes of their choice (see website). It is required that there
must be links between the research proposal and the activities of the
chosen institutes. The deadline is generally December. The candidates also
have the possibility to visit a Japanese University through the
Postdoctoral Program of the Japan Society for the Promotion of Science. See
www.ihes.fr/EPDI.


\medskip


Other websites:

http://www.researchresearch.com (search funding opportunities, mathematics)

http://www.lms.ac.uk/jobs (with links to other sites)

http://www.jobs.ac.uk

http://www.maths.lth.se/nordic/Euro-Math-Job.html

\medskip

Please add suggestions!

\medskip

Dugald Macpherson, June 2004
\end{document}